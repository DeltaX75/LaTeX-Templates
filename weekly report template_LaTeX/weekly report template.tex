\documentclass{article}
\usepackage[utf8]{inputenc} % 输入编码(如果您的文本是UTF-8编码的)
\usepackage{amsmath}  % 提供了更多的数学符号和命令
\usepackage{amssymb}  % 提供了各种数学符号
\usepackage{textcomp} % 提供额外的符号
\usepackage{indentfirst} % 第一行也缩进,可用\noindent控制不缩进
\usepackage{graphicx} % 用于插入图形
\usepackage{cite}
% \cite{ctex} % 中文宏包
%%%% 以下宏包似乎不必要
% \usepackage[T1]{fontenc} % 字体编码
% \usepackage[english]{babel} % 用于英文排版和特殊字符
% \usepackage{geometry} % 用于页面布局设置
%%%%
\title{Weekly Progress Report}
\author{Yuchen Sun}
\date{2024-01-01}

\begin{document}
\maketitle
\section*{Main Focus}

Intensive Reading - Consensus Problems in Networks of Agents With Switching Topology and Time-Delays.

\begin{itemize}
    \item Progress: I have read the abstract, Part I to III.
\end{itemize}

\section*{Insights}

\begin{itemize}
    \item Reading ability: This is the first time for me to intensive read a paper. I have learnt a number of words about automatic and math field.
    \item Knowledge Learnt: Basic Graph Theory Concepts; Relearning and the Application on Matrix Analysis;
    \item Developed Skills: Using ChatGPT to assist with reading paper.
\end{itemize}

\subsection*{Idea in the paper}

\noindent\textbf{Concept}

I have learned the concepts of "agree" and "reaching a consensus" and their difference:

\begin{quote}
We say nodes \( v_i \) and \( v_j \) \textit{agree} in a network if and only if \( x_i = x_j \).

We say nodes of a network have \textit{reached a consensus} if and only if \( x_i = x_j \) for all \( i, j \in \mathcal{I} \).
\end{quote}

Note: \textit{to agree} is a node-to-node status which focuses on the locality. While "\textit{to reach a consensus}" is a global status that must satisfy...

\section*{Reflections}
I gained some valuable insights from this week's learning process. First, when encountering unfamiliar terms, I've learned that continuing to read often leads to clarity, as the answer might be present later in the context. Secondly, I realized the need to bolster my coding skills; although I can identify and correct errors in programs written by GPT-4, I am not yet capable of coding independently.

\section*{Plans for Next Week}
\begin{itemize}
    \item Complete the neural network and deep learning course on Bilibili.
    \item Resolve the issue with the T-S neural network program.
    \item Continue reading and understanding the paper on \textit{"Consensus Problems in Networks of Agents With Switching Topology and Time-Delays."}
\end{itemize}
%%%% References
\bibliographystyle{unsrt}
\bibliography{reference} % display references
%%%%
\end{document}


